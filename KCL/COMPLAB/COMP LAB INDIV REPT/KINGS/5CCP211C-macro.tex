%%%%%%%%%%%%%%%%%%%%%%%%%%%%%%%%%%%%%%%%%%%%%%%%%%%%%%%%%%%
% _______ ______ __  __ _____  _            _______ ______ 
%|__   __|  ____|  \/  |  __ \| |        /\|__   __|  ____|
%   | |  | |__  | \  / | |__) | |       /  \  | |  | |__   
%   | |  |  __| | |\/| |  ___/| |      / /\ \ | |  |  __|  
%   | |  | |____| |  | | |    | |____ / ____ \| |  | |____ 
%   |_|  |______|_|  |_|_|    |______/_/    \_\_|  |______|
%%%%%%%%%%%%%%%%%%%%%%%%%%%%%%%%%%%%%%%%%%%%%%%%%%%%%%%%%%%%
\documentclass{article} % Don't change this
%
\usepackage[english]{babel}
\usepackage[utf8]{inputenc}
\usepackage[margin=1.5in]{geometry}
\usepackage{amsmath}
\usepackage{amsthm}
\usepackage{amsfonts}
\usepackage{amssymb}
\usepackage[usenames,dvipsnames]{xcolor}
\usepackage{graphicx}
\usepackage[siunitx]{circuitikz}
\usepackage{tikz}
\usepackage[colorinlistoftodos, color=orange!50]{todonotes}
\usepackage{hyperref}
\usepackage[numbers, square]{natbib}
\usepackage{fancybox}
\usepackage{epsfig}
\usepackage{soul}
\usepackage[framemethod=tikz]{mdframed}
%
\newcommand{\blah}{blah blah blah \dots}
%
\setlength{\marginparwidth}{3.4cm}
%To use symbols for footnotes
\renewcommand*{\thefootnote}{\fnsymbol{footnote}}
%%%%%%%%%%%%%%%%%%%%%%%%%%%%%%%%%%%%%%%%%%%%%%%%%%%%%%%%%%%
\title{
\normalfont \normalsize 
\textsc{5CCP211C-Introduction to numerical modelling, \\Physics Dept., King's College London} \\
[10pt] 
\rule{\linewidth}{0.5pt} \\[6pt] 
\huge Title of Computational Practical \\
\rule{\linewidth}{2pt}  \\[10pt]
}
\author{Full Name-Student 1; Full-Name-Student 2}
\date{\normalsize \today}

\begin{document}

\maketitle
\noindent
Date performed \dotfill January 0, 2018 \\
Group \dotfill 1 or 2 \\
Pair \dotfill Pair\# \\

\section{Before the lab.}
\label{Sec:before}
Report here the answer to the questions on the theoretical aspects of the practical itself. It might be to report a bit of physics, or an algorithm, or some coding.
Add an in-line equation use the dollar symbol, as $\left(x+1 = y\right)$, while if you want to number them and to be able to cite them in the text, as see Eq.\ref{Eq:lin}
\begin{equation}
x+1 = y  \mbox{~~~.}
\label{Eq:lin}
\end{equation}
Please use eqnarray if you want a bunch of equations, where all of them are numbered or only a few, as in Eq.\ref{Eq:lin2}
\begin{eqnarray}
\nonumber
x+1 &=& y \\  
y &=&  \frac{\partial x}{\partial t} \\
\nonumber x &\in& [0,\infty)   \mbox{~~~.}
\label{Eq:lin2}
\end{eqnarray}
Eqs.\ref{Eq:lin}-\ref{Eq:lin2} are not to be meaningful of any specific physical problem.

\section{Plan of your code}
You should report here the structure of your code and how it was implemented.
Make a good use of what you have in Sec. \ref{Sec:before}, and try to use meaningful name =s for your variables.

\section {Collection of Data}
If it is required to collected data, report what simulations have you performed and sum-up their results. Pls if you upload a image, be sure that it is properly numbered and captioned. In the text, refer to it when needed.

%%%%%%%%%%%%%%%%%%%%%%%
% FOR A NUMBERED LIST
% \begin{enumerate}
% \item Your_Item
% \end{enumerate}
%%%%%%%%%%%%%%%%%%%%%%%
% FOR A BULLETED LIST
% \begin{itemize}
% \item Your_Item
% \end{itemize}
%%%%%%%%%%%%%%%%%%%%%%%

%%%%%%%%%%%%%%%%%%%%%%%%%%%%%%
% TO IMPORT AN IMAGE
% UPLOAD IT FIRST (HIT THE PROJECT BUTTON TO SHOW FILES)
% KEEP THE NAME SHORT WITH NO SPACES!
% TYPE THE FOLLOWING WITH THE NAME OF YOUR FILE
% DON'T INCLUDE THE FILE EXTENSION
% \includegraphics[width=\textwidth]{name_of_file}
% \textwidth makes the picture the width of the paragraphs
%%%%%%%%%%%%%%%%%%%%%%%%%%%%%%
% TO CREATE A FIGURE WITH A NUMBER AND CAPTION
% \begin{figure}
% \includegraphics[width=\textwidth]{image}
% \caption{Your Caption Goes Here}
% \label{your_label}
% \end{figure}
% REFER TO YOUR FIGURE LATER WITH
% \ref{your_label}
% LABELS NEED TO BE ONE WORD
%%%%%%%%%%%%%%%%%%%%%%%%%%%%%

\section {Discussion of Data}
%%%%%%%%%%%%%%%%%%%%%%%%%%%%%%%%%%%%%%%%%%%%%%%%
Make a short statement of what you have learned, and which information you have gathered, and how your code works towards it.

\section{Constructive Feedback}
Not mandatory but you can suggest improvements for future labs.

%%%%%%%%%%%%%%%%%%%%%%%%%%%%%%%%%%%%%%%%%%%%%%%%%%%%%%

			% BIBLIOGRAPHY: %
% Make sure your class *.bib file is uploaded to this project by clicking the project button > add files. Change 'sample' below to the name of your file without the .bib extension.
%%%%%%%%%%%%%%%%%%%%%%%%%%%%%%%%%%%%%%%%%%%%%%%%%%

%\bibliographystyle{plainnat}
%\bibliography{sample}

% UNCOMMENT THE TWO LINES ABOVE TO ENABLE BIBLIOGRAPHY

%%%%%%%%%%%%%%%%%%%%%%%%%%%%%%%%%%%%%%%%%%%%%%%%%%

\end{document} 